\documentclass[12pt]{book}
\usepackage{ amssymb }
\newcommand{\Lag}{\mathcal{L}}
\begin{document}
\chapter{Consumption Model}

\section{Objective}

This document describes the consumption model of the economics engine.

\section{Motivation}

POP consumption is, with decay and war damage, one of the great
resource sinks of the economics engine. Unlike the other two, it is
also the purpose of having a detailed economics engine in the first
place. 

\section{Goals}

The consumption model has several desiderata:
\begin{itemize}
\item Allow POPs to trade off one consumption against another along
  utility frontiers, e.g. apples and oranges - if one apple and one
  orange give satisfaction, then so should three apples or three
  oranges, or any combination along the implied curve of these three
  points.
\item Allow both prioritisation and parallel consumption. A POP faced
  with the choice of doing without food, or doing without shelter,
  should choose the latter; but a POP may still use shelter, if it is
  available, if there is no food, and derive some benefit from doing
  so. A POP with a house but no food is still better off than the one
  who lacks both.
\item Give POPs a time preference, so that they cease consuming when
  the marginal utility of doing so is outweighed by the net present
  value of having that money for future turns - not, for example, when
  they run out of money (unless they are close to subsistence). In
  particular, it should be possible for time preferences to vary, both
  intrinsically and with the available investment opportunities.
\end{itemize}

\section{Model}

POPs begin a turn with several irritants, for example hunger, cold,
and boredom. (Some irritants may vary with external inputs, like cold;
others are constant.) Each irritant has a weight to give its relievers
a priority; thus hunger may have a higher weight than cold, and
accordingly food will initially have a higher priority than shelter -
but not infinitely higher; a POP that has eaten something but not all
the available food may start looking for clothes. Each irritant
contributes to \emph{fragility}, a measure of how vulnerable the POP
is to endemic diseases and general malaise.

The irritants are reduced, possibly to zero, by \emph{consumption
  baskets}, collections of resources to be used. Within each basket,
some resources are substitutable for others, such as apples and
oranges; using only one of the options to meet that part of the basket
will require more of the resource than the exchange rate at the
midpoint would imply - three oranges or three apples, versus an orange
plus an apple, for example. Additionally, each basket may define a
nondiversity penalty, a separate irritant with its own fragility
contribution, to be removed by consuming that basket in a diverse
way. 

Like industrial inputs, baskets may consist of consumables, movable
capital, and fixed capital. POPs will seek to consume the basket that
reduces their fragility the most; if they cannot buy all the goods
that one requires, or if the cost of doing so is too high, they will
attempt the next basket, and so on until they either consume one or
run out of baskets (and presumably starve to death). 

\section{Math}

Given two substitutable goods, there is an indifference curve where
any point on the curve satisfies the basket. Note that this cannot be
linear, or even have linear parts, because for prices $p_x$, $p_y$,
the cheapest point on a linear substitution curve is always at an
endpoint. (If you move a distance $\delta x, \delta y$, you change the
price by the same amount $p_x\delta x + p_y \delta y$ no matter where
on the curve you are, so you always slide all the way to the end in
the direction of negative price change). So, we need an inverse
relationship, which we can parametrise fairly freely:
\begin{eqnarray}
(ax + o)^n = \frac{D^2}{(by +o)^n}
\end{eqnarray}
where $o$ is an offset showing how far the asymptotes are from the
axes. Since we need to be able to reach the axes, i.e. have 0 of one
or the other good, the offset must be positive, and it is the same on
both sides of the equation to maintain symmetry. The constants $a$ and
$b$ are substitution weights, and the distance $D$ gives the minimum
distance from the origin, as shown below. We see that $n$ does not
matter in this equation no matter what we're solving for, since we can take the
$n$-th root of the equation and then just make the substitution
$D^2\to D^{2/n}$. This leaves four parameters ($a,b,D,o$) to be found
by the two constraints that the curve must pass through $(x_0,0)$ and
$(0, y_0)$; here $x_0$ and $y_0$ have the intuitively clear meaning of
being the amount of $x$ and $y$ to be consumed in the absence of the
other resource, which is easy for a user to visualise and specify.
We therefore take the offset, distance, and sole-resource amounts as
to be specified, and solve for the substitution weights:
\begin{eqnarray*}
(ax + o)(by + o) &=& D^2\\
(ax_0 + o)o &=& D^2 = o(by_0 + o)\\
ax_0 &=& \frac{D^2}{o} - o = by_0\\
a &=& \frac{D^2 - o^2}{ox_0}\\
b &=& \frac{D^2 - o^2}{oy_0}
\end{eqnarray*}
We note that, if $x_0=y_0$, then $a=b$ and, given equal prices
$p_x=p_y$, the cheapest point occurs at $(D, D)$ - which is also the
closest point to the origin, hence labeling this quantity the
`distance'. It could also be called the ``(half) minimum amount'', since with
different prices that move the cheapest point away from $(D, D)$, the
unweighted amount of resources to be consumed will be larger than the
$2D$ consumed at $(D, D)$ - for example, if the prices are such as to
make $x=2D<x_0$, $y$ will still be larger than zero.

Extending to three or more dimensions is straightforward:
\begin{eqnarray*}
\prod_{i=i}^{n}(c_ix_i + o) &=& D^2\\
c_ix_{i0} + o &=& \frac{D^2}{o^{n-1}}\\
c_i &=& \frac{D^2-o^n}{o^{n-1}x_{i0}}.
\end{eqnarray*}

Now, given prices $p_x$ and $p_y$, we can determine the cheapest point
on the curve:
\begin{eqnarray*}
P &=& p_xx + p_yy\\
\frac{\partial P}{\partial x} = 0 &=& p_x + p_y\frac{\partial y}{\partial x}\\
-p_x &=& p_y\frac{\partial }{\partial x}\left(\frac{D^2/b}{ax+o}-\frac{o}{b}\right)\\
-p_x &=& \frac{p_yD^2}{b}\frac{\partial }{\partial x}\left(\frac{1}{ax+o}\right)\\
p_x &=& \frac{p_yD^2}{b}\frac{a}{(ax+o)^2}\\
(ax+o)^2 &=& \frac{ap_yD^2}{bp_x}\\
a^2x^2 + 2axo + o^2 - \frac{ap_yD^2}{bp_x} &=& 0 \\
ax^2 + 2xo + o^2 - \frac{p_yD^2}{bp_x} &=& 0 \\
x &=& \frac{-2o\pm\sqrt{4o^2 - 4(o^2 - C)}}{2a} \\
x &=& \frac{-o\pm\sqrt{C}}{a}
\end{eqnarray*}
where the quantity $C = \frac{p_yD^2}{bp_x}$ has been
introduced for brevity. We see that $x$ is increasing in the price of
$y$ and decreasing in its own price, which is reasonable. Extending to
three dimensions is not entirely trivial by the substitution method
above; it requires evaluating six partial derivatives, and although
the symmetry of the problem let us find one and get the other five by
permuting the subscripts, that's a lot of subscripts to keep track
of. Instead we use the Lagrangian multiplier. We have the cost
function
\begin{equation}
P(x,y,z) = p_xx+p_yy+p_zz
\end{equation}
to be minimised subject to the constraint
\begin{equation}
(ax+o)(by+o)(cz+o) = D^2
\end{equation}
giving the Lagrangian
\begin{equation}
\Lag(x,y,z,\lambda} = p_xx+p_yy+p_zz -
\lambda\left((ax+o)(by+o)(cz+o)-D^2\right).
\end{equation}
Take the gradient and set it to zero:
\begin{eqnarray*}
p_x - a\lambda(by+o)(cz+o) &=& 0\\
p_y - b\lambda(ax+o)(cz+o) &=& 0\\
p_z - c\lambda(ax+o)(by+o) &=& 0\\
(ax+o)(by+o)(cz+o)-D^2     &=& 0.
\end{eqnarray*}
Introduce for brevity $X=ax+o$, $Y=by+o$, $Z=cz+o$, and start solving:
\begin{eqnarray*}
a\lambda YZ &=& p_x \\
b\lambda XZ &=& p_y \\
c\lambda XY &=& p_z \\
XYZ         &=& D^2 \\
\\
Z             &=& \frac{D^2}{XY} \\
a\lambda YD^2 &=& XYp_x \\
b\lambda XD^2 &=& XYp_y \\
c\lambda XY   &=& p_z \\
\\
Z                        &=& \frac{D^2}{XY} \\
a\lambda \frac{D^2}{p_x}  &=& X \\
b\lambda \frac{D^2}{p_y}  &=& Y \\
c\lambda XY              &=& p_z \\
\\
Z                        &=& \frac{D^2}{XY} \\
X                        &=& a\lambda \frac{D^2}{p_x} \\
Y                        &=& b\lambda \frac{D^2}{p_y} \\
abcD^4\lambda^3          &=& p_xp_yp_z \\
\\
Z                        &=& \frac{D^2}{XY} \\
X                        &=& a\lambda \frac{D^2}{p_x} \\
Y                        &=& b\lambda \frac{D^2}{p_y} \\
\lambda                  &=& \left(\frac{p_xp_yp_z}{abcD^4}\right)^{1/3}
\\
X                        &=& a\left(\frac{p_xp_yp_z}{abcD^4}\right)^{1/3} \frac{D^2}{p_x} \\
Y                        &=& b\left(\frac{p_xp_yp_z}{abcD^4}\right)^{1/3} \frac{D^2}{p_y} \\
Z                        &=& \frac{D^2}{XY} \\
\\
X                        &=& \left(\frac{a^2p_yp_zD^2}{bcp_x^2}\right)^{1/3}  \\
Y                        &=& \left(\frac{b^2p_xp_zD^2}{acp_y^2}\right)^{1/3}  \\
Z                        &=& \frac{D^2}{XY}\\
\\
X                        &=& \left(\frac{a^2D^2}{bc}\frac{p_yp_z}{p_x^2}\right)^{1/3} \\
Y                        &=& \left(\frac{b^2D^2}{ac}\frac{p_xp_z}{p_y^2}\right)^{1/3} \\
Z                        &=& \left(\frac{c^2D^2}{ab\frac{p_z^2}{p_xp_y}}\right)^{1/3} \\
\\
X                        &=& \left(\frac{a^2D^2}{bc}\frac{p_yp_z}{p_x^2}\right)^{1/3}  \\
Y                        &=& \left(\frac{b^2D^2}{ac}\frac{p_xp_z}{p_y^2}\right)^{1/3}  \\
Z                        &=& \left(\frac{c^2D^2}{ab}\frac{p_xp_y}{p_z^2}\right)^{1/3}.
\end{eqnarray*}
We see that each of $X, Y, Z$ decreases in its own price and increases
in the other two, as we would expect, and that the symmetry is preserved.
The extension to four dimensions is straightforward but tedious.








\end{document}




