\documentclass[12pt]{book}
\title{Guiding Spirits design notes}
\begin{document}
\maketitle
\section{Overall Idea}

``Guiding Spirits'' (working title) is a grand strategy game in which you play as the
inspiring spirit of a region, tribe, or ideology rather than as a
particular state or country. The fundamental action of the game is to
inspire a simulated population unit, or pop, to do something it
would not ordinarily do, such as building a factory or declaring
war.

The player may influence any pop anywhere in the world; however, he
has a limited budget of inspiration points (IP), and the cost of
influence varies with the pop. It is affected by:

\begin{itemize}
\item Distance from the player's location.
\item The player's choice of spiritual attributes; for example, he may
  choose to play an ethnic spirit which easily influences pops of a
  particular nationality, a regional spirit which is powerful in a
  particular geographical location, or an idea spirit which finds it
  cheaper to influence pops to particular actions.
\item Whether other spirits have influence over the pop.
\item Difficulty or cost of the action for the pop.
\end{itemize}

The player's score is the wealth and happiness of all pops in the
world, weighted by how much influence the player's spirit has over
them; thus a regional spirit will do best to maximise the wealth
contained in its region, while an ideological spirit may want to
maximise its influence in the world, and secondarily global wealth.

The pops organise themselves into tribes and nations, kingdoms and
republics; they produce, buy, and sell goods in markets; they muster
into armies and invade their neighbours. All these actions may be
decided upon by the pops themselves, or spurred by the player spirits'
expenditure of IP. Note that although the player may find it
convenient to create a large kingdom whose ruler he has much
influence over, this is only one style of play. An ideological spirit
may prefer to spread a religion instead, without bothering too much
about how its adherents organise themselves into polities, as long as
each statelet is ready to fight for the faith. For several players,
therefore, one can set up a traditional grand strategy game by playing
regional or national spirits, and attempting to maximise the wealth
and power of the chosen people; but one could also play a cooperative
game by having spirits whose areas of responsibility complement each
other - or play as competing religions rather than competing nations.

\section{Gameplay}

The game is played on a board consisting of provinces, on which pops
may move and work. 


\end{document}
