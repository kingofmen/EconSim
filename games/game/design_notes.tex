\documentclass[12pt]{book}
\begin{document}
\chapter{Production}

\emph{This section must be basically rewritten as I changed how it
  works, abstracting installation time entirely into the cost}.

Some production variants have \verb|fixed_capital|, goods attached to the
Field object. These are intended to model, for example, heavy
machinery, drainage systems, or fences: Enabling tools not easily
moved, and usually taking considerable effort to install or
create. Generally such variants will require less labour or other
input. Fixed capital - sometimes 'fixcap', as opposed to movable
capital or 'mobcap' - is installed in the 'capex' phase of production
step execution. The installation cost is stored in the
\verb|capex_cost| field of the Input; this is the amount it costs to
create one full scale's worth of fixcap. Lesser amounts, as for
maintenance, scale proportionally. In addition, installing fixcap
reduces the efficiency of the process as a whole. There are two ways
one could model this: First, suppose that in seven days the process
builds seven units, one per day. Then taking one day to install
capital will reduce the output to six - still one per day. On the
other hand, suppose that in seven days the process outputs seven
units, each of which took the whole seven days to make; then taking a
day for fixcap will give you seven units each six-sevenths complete,
which is not at all the same thing as six units. 

I deal with this by completely ignoring the parallel-production issue;
production is assumed to be so granular that it is exactly linear in
the time available. Production is thus modelled as an area, time on
the X axis, working space on the Y axis. Capex for a unit of working
space takes up some of the time available, reducing the amount
produced by that working space. So, if you have half a unit of fixcap,
and you add another fourth of a unit at the price of half the
available time, then you will produce half a scale, plus one-fourth of
a scale times one-half. The time cost of capex install is given by 
\verb|capex_time|, while the amount of fixcap installed for a specific
Progress is given by \verb|installed_fixcap_u|.

\chapter{Transport}

An Area models a collection of economic activity whose logistics can
reasonably be abstracted into production costs; for example, a city
and its immediate hinterland. The farmers, clearly, are expending some
labour and using some capital to bring their produce to market; but
for purposes of the economic model, this is taken as part of the cost
of producing the grain and fruit in the first place. When the area
grows large enough that this
abstraction becomes unreasonable, we split economic activity into two
Areas and explicitly model transport, logistics, and infrastructure
between them, in the shape of a Connection.

A Connection models a set of routes between two points which are
sufficiently close together and interconnected to be considered as one
logistical or tactical unit. If the area between two cities is an
obstacle-less plain, across which humans may wander at will in any
direction, it can be considered a single Connection even if there are
a large number of routes they may take. Conversely, if the cities are
separated by a mountain range, then each pass through that range,
however close together as the lammergeier flies, is a
separate Connection, because there is no easy way of switching from
one route to another.

In addition to the distance between the endpoints, a Connection also
models a width, roughly speaking the amount of space a unit has to
maneuver laterally. A mountain pass has a small amount of width, an
open plain has a lot, the Atlantic is the epitome of width. One can
think of width as the value of mobility and maneuverability, or as how
much choice an army has in whether to accept battle at a defending
army's chosen spot. If an army marches through the Brenner Pass, and is
opposed, it must either engage the defending army head-on, or give up
reaching Italy. On the other hand, an army seeking to reach Warsaw can
do a lot of probing for flanks or attempting to steal a march, and is
certainly under no obligation to engage at any particular point; just
go fifteen miles north and head for Warsaw again; the defenders can
sit in their chosen spot and watch you go, or they can march north to
match you.

In addition to width, the amount of Infrastructure in a Connection
also matters for the importance of maneuver. If we think of two US
freeways close enough together to form a single Connection, it will be
a lot easier to switch forces from one to the other in Kansas, where
there's plenty of lateral back roads connecting the two, than in the
Rocky Mountains. The Infrastructure, then, represents internal
communications in the Connection - lateral roads, railroad spurs,
small regional airports, canals connecting rivers. The more of it
there is, the more easily troops can maneuver within the Connection,
and the more important the general's skill and the troops' mobility
becomes.

Connections also have a ruggedness, affecting the speed at which units
may move through it, a terrain type, such as marsh or forest, and a
number of defensible points. They may specify their local weather and
similar effects such as icebergs. Connections may be raided from
specified locations with a given range; for example a Mid-Atlantic
route might be vulnerable to naval units based at Brest and at various
British harbours, but not at Toulon. Similarly a Rhine connection
might be vulnerable to any number of hilltop castles.








\end{document}
